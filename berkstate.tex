\documentclass[10pt]{article}
\usepackage{fullpage}


\usepackage{color,hyperref}
\definecolor{darkblue}{rgb}{0.0,0.0,0.3}
\hypersetup{colorlinks,breaklinks,
            linkcolor=darkblue,urlcolor=darkblue,
            anchorcolor=darkblue,citecolor=darkblue}


\begin{document}
\setlength\parindent{0pt}

Don Blair\\
Fellowship Application \\
Berkman Center for Internet and Society \\
12/12/2014

%Reference:  civic team: https://civic.mit.edu/team

\section*{Motivation, and Statement of Work}

Over the last few years, I've begun to connect with online communities around the practice of 'open science'.  At first, my focus was merely finding ways of importing the open source ethos I'd found in online hobby and software communities into the academic research laboratories with which I was familiar; but after discovering groups like \href{http://publiclab.org}{Public Lab}, my goals have shifted:  I am eager to help build new socio-technical structures that might help to support more open, networked, and democratic forms of science and engineering, with an agenda that is focused on the most urgent problems facing communities around the world today.  Although my recent work has been on the 'environmental monitoring' and 'environmental justice' issues that have arisen within the Public Lab community, I am also increasingly excited by the constellation of possibilities and solutions that arise within 'open source agriculture' -- addressing issues around {\bf water, food, energy, and social connection} in novel ways.  I have found my recent discussions with people from the Berkman community (in their travels to the Center for Civic Media, and elsewhere) particularly inspiring and productive; I view this Fellowship as an opportunity to formalize and solidify my connections with the Berkman community, as well as to provide a base upon which to begin an array of collaborative projects.  My current thinking (though it is constantly evolving -- every conversation at Berkman has sparked new ideas!) is that I might focus my efforts on the following areas:


%The aim here is to build on initiatives already underway within the Civic community by connecting them to complementary intiatives in the Public Lab and FarmHack communities.  Ongoing and planned Civic initiatives focused on civic monitoring projects in the US, Brasil, Kenya, China and elsewhere might usefully draw upon the technical and social support structures for community science being developed within Public Lab's Open Water, Open Air, and Open Land initiatives; indeed, there is already work being done on the Open Water project at the Center.  FarmHack's Growing Clean Water adds a long-term remediation perspective, and the community around food production, to this collaborative space.  My aim would be to leverage the expertise and talent in Civic to convene collaborations around the design processes, community education practices, and various notions of accessibility that undergird these projects.  I look forward to continuing to work with Heather Craig on this project, and see strong overlaps with the work being done by Emilie, Wang Yu, Jude Mwenda Ntabathia, Erhardt, and Nathon in this area.

\paragraph{The role of institutions, hierarchy, and authority in cooperative communities.} As communities gain access to more powerful tools for decentralized monitoring, manufacturing, energy production, and remediation, and as existing institutions seek to accommodate these new sources of data and material output, important issues of regulation, certification, calibration, data veracity, and safety are increasingly coming to the fore.  Are informal, decentralized, collaborative communities adequately prepared to deal with the questions of power, hierarchy, and authority that arise in the face of such transformations? I would like to explore these issues 'in situ' as they emerge within a set of ongoing, concrete projects in the areas of: 

\begin{itemize}
\item environmental monitoring (\href{http://publiclab.org}{Public Lab});
\item remediation, sanitation, and food production (\href{http://farmhack.net}{Farm Hack}); and 
\item human health (the MIT \href{http://breastpump.media.mit.edu/}{Breast Pump Hackathon}; other civic health technologies)
\end{itemize}

Such projects might serve as important 'laboratories' within which to explore novel frameworks and analyses (via the production of essays and working papers) and around which to generate and facilitate ongoing dialogues (via meet-ups and colloquia); examples of questions that have already emerged include: 

\begin{itemize}
\item Which DIY technologies ought to be regulated or prohibited? Through what mechanisms?
\item Are decentralized certification schemes effective? What precedents and current practices might inform such schemes?
\item What balance ought to be struck between openness and privacy in projects that promote civic environmental and health data reporting? 
\item Might we be able to develop a ``A Legal Toolkit for Open Community Science and DIY Environmental Monitoring and Health''
\end{itemize}

%I look forward to working  Erhardt Graeff, J. Nathan Matthias, Tal Achituv (Fluid Interfaces), Willow Brugh, and Alexis Hope, and others

\paragraph{Civic engagement through narrative.} Community science projects require careful consideration of 'narrative', and `narrative technique':  how is the story of a project -- its goals, rationale, and history -- going to be conveyed to participants? How will its outcomes be communicated to the various intended audiences of the project? For these important questions, the traditional methods of the professional and academic fields whose focus is on narrative, rhetoric, and interpretation - journalism, communications, literature, design, and the humanities - contain many useful insights.  Recently, Catherine D'Ignazio of Emerson College, in collaboration with the Center for Civic Media, Public Lab, and other institutional partners, has been building out a vision for an \emph{Environmental Storytelling Institute} whose aim is to create mutual learning opportunities among scientists, journalists, and local residents impacted by environmental issues.  The \emph{ESI} would include a series of workshops and meet-ups in which the participants would share communication techniques while working on output in various media (articles, blog posts, artworks) and learning about the relevant science.  I would hope to use my appointment as an Affiliate as an opportunity to work with the \emph{ESI} initiative and to connect it to ongoing projects and discussions at Berkman -- specifically, the `Networks Story' group led by Dalia Othman.

%Catherine's work, the networked stories group, the Future of News Initiative, Matt, Adrienne, Rahul Bhargava. 

\paragraph{Digital Labor.} My experience in the last few years of working as a volunteer within the Public Lab and Farm Hack communities has led to a strong personal interest in reimagining the ways in which material support might be better distributed to open source developers and researchers. Discussing works like David Graeber's ``Debt: The Last 5000 Years'' with friends at Berkman and the Center for Civic Media at MIT has led to promising ideas; I'm eager to explore these ideas further. 

\paragraph{Other research areas.} Finally, I hope with this appointment to have the opportunity to co-write several extended essays whose content will stem from ongoing conversations that have begun with researchers at Berkman and at Civic.  Recent topics (among the many possible) include: 

\begin{description}

\item[``Hidden Needs.''] One of the important outcomes of the Breast Pump Hackathon was the recognition that an important and ubiquitous health care need has been insufficiently addressed by designers and technologists for decades. How had this oversight persisted for so long, and how was it eventually exposed?  Are there ways of facilitating - through the use of the tools and techniques of civic engagement - the exposure of the 'hidden needs' among marginalized groups?  How do we design tools for groups whose marginalization, and associated group identity, is episodic and temporary - as is the case with nursing mothers and infants, students, prisoners, tourists, hospital visitors, and others?

\item[``Dangerous Toys.''] The widespread adoption of novel and potentially useful civic technologies - ranging from drones, to DIY medical devices - has in many cases been slowed or prevented by concerns over their misapplication; often, regulations and precedents that address adjacent, but substantially different, technologies often seem to be cited from a perspective that is overly conservative, or inconsistent, when attempting to strike a useful balance between the risk and benefit to their users. How well-grounded in empirical reality are these 'arguments from harm' for various civic technologies?  This project would examining the various degrees of regulation (extent and scope of the relevant legal corpus) of civic technologies across different legal spheres and/or across cultures, relating this to statistics on the risks to users (accident statistics), with the goal of illuminating cultural assumptions about the proper use of civic technologies. 

%\item[``Social Network Regulation?''] Public health and social psycholgy researchers have accumulated substantial evidence for a positive relationship between the strength and structure of an individual's social network, and various measures of well-being.  What 

\end{description}

I very much look forward to the prospect of engaging in these and other projects at Berkman!


\end{document}

