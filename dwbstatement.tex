\documentclass[10pt]{article}
\usepackage{fullpage}


\usepackage{color,hyperref}
\definecolor{darkblue}{rgb}{0.0,0.0,0.3}
\hypersetup{colorlinks,breaklinks,
            linkcolor=darkblue,urlcolor=darkblue,
            anchorcolor=darkblue,citecolor=darkblue}


\begin{document}
\setlength\parindent{0pt}

Don Blair\\
Research Affiliate Appointment \\
MIT Center for Civic Media \\
11/4/2014

%Reference:  civic team: https://civic.mit.edu/team

\section*{Statement of Work}


As a Research Affiliate of the Center for Civic Media, I would hope to contribute to an array of existing and emerging projects and themes (detailed below) at the Center, and in several modes: through the writing of blog posts, working papers, and book chapters; by assisting others in their projects (through technical development and strategizing); by engaging in and facilitating dialogue among members of the Civic community and the other communities I have worked with in recent years.  Below is an initial project list  (surely to undergo revision over time):

\paragraph{Support structures for community science.} In this area, my aim would be to connect the citizen monitoring initiatives already underway in the Civic community - in the US, Brasil, Kenya, China, and elsewhere -  to complementary projects in the \href{http://publiclab.org}{Public Lab}, and \href{http://FarmHack}{FarmHack} communities.  Public Lab's \emph{Open Water} project has already emerged as a collaborative project within the Civic Media Community; Public Lab's \emph{Open Air} and \emph{Open Land} initiatives contain similar overlaps with current and planned Civic Media projects, and my hope would be to develop these connections further. FarmHack's \emph{Growing Clean Water}, with its focus on environmental monitoring and remediation in the context of agriculture and land management, represents an additional,  important potential focus for Civic's research efforts: as the world's population grows, the environmental impacts and resource needs of food production will become even more tightly coupled to the daily concerns of civic communities worldwide.  In the course of establishing and deepending these connections, my hope would be to leverage the expertise and talent that has convened within the Civic community in exploring improved and more accessible civic technology designs and novel community education practices in support of the `community-' or `civic-science' practices that are emerging in these areas.  

%The aim here is to build on initiatives already underway within the Civic community by connecting them to complementary intiatives in the Public Lab and FarmHack communities.  Ongoing and planned Civic initiatives focused on civic monitoring projects in the US, Brasil, Kenya, China and elsewhere might usefully draw upon the technical and social support structures for community science being developed within Public Lab's Open Water, Open Air, and Open Land initiatives; indeed, there is already work being done on the Open Water project at the Center.  FarmHack's Growing Clean Water adds a long-term remediation perspective, and the community around food production, to this collaborative space.  My aim would be to leverage the expertise and talent in Civic to convene collaborations around the design processes, community education practices, and various notions of accessibility that undergird these projects.  I look forward to continuing to work with Heather Craig on this project, and see strong overlaps with the work being done by Emilie, Wang Yu, Jude Mwenda Ntabathia, Erhardt, and Nathon in this area.

\paragraph{The role of institutions, hierarchy, and authority in cooperative communities.} As communities gain access to more powerful tools for decentralized monitoring, manufacturing, energy production, and remediation, and as existing institutions seek to accommodate these new sources of data and material output, issues of regulation, certification, calibration, data veracity, and safety are increasingly coming to the fore.  Further, important issues of power, hierarchy, and authority inevitably emerge when such powerful and transformative civic technologies are used by heterogeneous and distributed, collaborative communities.  I would like to explore these issues 'in situ' as they emerge within a set of ongoing, concrete projects in the areas of: 

\begin{itemize}
\item environmental monitoring (\href{http://publiclab.org}{Public Lab});
\item remediation, sanitation, and food production (\href{http://farmhack.net}{Farm Hack}); and 
\item human health (the MIT \href{http://breastpump.media.mit.edu/}{Breast Pump Hackathon}; other civic health technologies)
\end{itemize}

Projects within these focus areas might serve as important 'laboratories' within which to explore novel frameworks and analyses (via the production of essays and working papers) and generate and facilitate ongoing community dialogues (via meet-ups and colloquia); examples of questions that have already emerged in these areas which might be usefully addressed by members of the Civic community include:

\begin{itemize}
\item Which DIY technologies ought to be regulated or prohibited? Through what mechanisms?
\item Are decentralized certification schemes effective? What precedents and current practices might inform such schemes?
\item What balance ought to be struck between openness and privacy in projects that promote civic environmental and health data reporting? 
\end{itemize}

The gathered expertise and infrastructure at the Center would be a great boon in carrying out these and related projects. 

%I look forward to working  Erhardt Graeff, J. Nathan Matthias, Tal Achituv (Fluid Interfaces), Willow Brugh, and Alexis Hope, and others

\paragraph{Civic engagement through narrative.} In the course of my own work, and through my recent exposure to the vision of researchers affiliated with Civic Media, I have become convinced that many of the goals of novel community science communities like Public Lab migh usefully be addressed by using the traditional tools, methods, and approaches that have long been employed within professional and academic fields whose focus is on narrative, rhetoric, interpretation, and the art of communicating with specific audiences:  to wit, the fields of journalism, communications, literature, design, art, and many areas of the humanities.  Recently, Catherine D'Ignazio of Emerson College and I, in collaboration with the Center for Civic Media, Public Lab, and additional institutional partners, have been building out a vision for an \emph{Environmental Storytelling Institute}, whose aim is to create dialogue and mutual learning opportunities among scientists, journalists, and local residents - a series of workshops, discussions, and project in which the participants learn more effective ways of communicating environmental research to their target audiences, while also becoming more conversant in local environmental issues and the relevant science. I would hope to use my appointment as an Affiliate as an opportunity to work on this and other projects in order to find new ways of enhancing the effectiveness of narrative in the civic engagement context. 

%Catherine's work, the networked stories group, the Future of News Initiative, Matt, Adrienne, Rahul Bhargava. 

\paragraph{Other research areas} Finally, I hope with this appointment to have the opportunity to co-write several extended essays whose content will stem from ongoing conversations that have begun with researchers at Civic.  Recent topics (among the many possible) include: 

\begin{description}

\item[``Hidden Needs.''] With Catherine D'Ignazio, Tal Achituv, Alexis Hope, and others.  One of the important outcomes of the Breast Pump Hackathon was the recognition that an important and ubiquitous health care need has been insufficiently addressed by designers and technologists for decades. How had this oversight persisted for so long, and how was it eventually exposed?  Are there ways of facilitating - through the use of the tools and techniques of civic engagement - the exposure of the 'hidden needs' among marginalized groups?  How do we design tools for groups whose marginalization, and associated group identity, is episodic and temporary - as is the case with nursing mothers and infants, students, prisoners, tourists, hospital visitors, and others?

\item[``Dangerous Toys.''] The widespread adoption of novel and potentially useful civic technologies - ranging from drones, to DIY medical devices - has in many cases been slowed or prevented by concerns over their misapplication; often, regulations and precedents that address adjacent, but substantially different, technologies seem to be cited from a perspective that is overly conservative, or inconsistent, when attempting to strike a useful balance between the risk and benefit to their users. How well-grounded in empirical reality are these 'arguments from harm', for various technologies?  This project would examining the various degrees of regulation (extent and scope of the relevant legal corpus, say) on civic technologies across different legal spheres and/or across cultures, and compare these degrees of regulation against related statistics on various aspects of user risk, with the goal of generating some insights into how we might ameliorate our approach to calculating 'risk vs reward' when developing and deploying such civic technologies.  

%\item[``Social Network Regulation?''] Public health and social psycholgy researchers have accumulated substantial evidence for a positive relationship between the strength and structure of an individual's social network, and various measures of well-being.  What 

\end{description}

I very much look forward to the prospect of engaging in these and other projects at the Center for Civic Media.


\end{document}

